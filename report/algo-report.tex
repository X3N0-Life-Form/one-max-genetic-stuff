\documentclass[a4paper,10pt]{report}
\usepackage[utf8]{inputenc}
\usepackage[T1]{fontenc}
\usepackage{graphicx}

% Title Page
\title{Algorithmes adaptatifs}
\author{Adrien DROGUET}


\begin{document}
\maketitle

%\begin{abstract}
%\end{abstract}

\chapter{Algorithmes génétiques}
\section{Variables d'exécution}
\begin{itemize}
  \item \textbf{pop\_size} : Taille de la population, par défaut 200.
  \item \textbf{numberOfIterations} : Nombre d'itérations de l'algorithme avant terminaison, par défaut 300.
  \item \textbf{k} : Nombre de flips à faire lors d'un k-flip, par défaut 3.
  \item \textbf{t} : Nombre d'individus d'un tournoi, par défaut 5.
\end{itemize}



\section{Algorithmes disponibles}
\subsection{Sélection}
Par défaut: \textit{tournament}.
\begin{itemize}
  \item \textbf{Best} : Choisit les deux meilleurs instances.
  \item \textbf{Random} : Choisit deux instances aléatoires.
  \item \textbf{Worst-Best} : Choisit la meilleure et la pire instance.
  \item \textbf{Tournament} : Sélectionne les deux meilleures instances d'une sous-population.
\end{itemize}


\subsection{Croisement}
Par défaut: \textit{cross point}
\begin{itemize}
  \item \textbf{Cross Point} : Coupe deux instances à un point aléatoire.
  \item \textbf{Cross Uniform} : Mélange le contenu de deux instances.
\end{itemize}


\subsection{Mutation}
Par défaut: \textit{Bit Flip}
\begin{itemize}
   \item \textbf{Bit Flip} : Chaque bit d'une instance a 1/n chances d'être inversé.
   \item \textbf{K Flip} : Inverse k bits d'une instance.
\end{itemize}


\subsection{Insertion}
Par défaut: \textit{Replace Worst}
\begin{itemize}
  \item \textbf{Compare with parents} : Compare et (potentiellement) remplace les parents des instances générées.
  \item \textbf{Age} : Remplace les plus vieilles instances d'abord.
  \item \textbf{Replace Worst} : Remplace les pires instances.
\end{itemize}


\pagebreak
\section{Tests et résultats}
Chaque graphique représente la moyenne de 10 lancement pour la configuration décrite.
On notera que certaines configurations présentant peu d'intérêt ont été ignorées.

\subsection{Configuration par défaut}
Rappel: notre configurations par défaut utilise un sélection de type tournoi de taille 5, sur un population
de 200 instances de 10 bits, et itère 300 fois en utilisant des mutations de type bit-flip, avec croisement par
point et remplacement des pires instances.

\begin{figure}[h]
  \begin{center}
    \includegraphics[width=320px]{images/graph-default.png}
  \end{center}
  \caption{tournament - cross point - bit-flip - replace worst}
\end{figure}

\pagebreak
\subsection{Variation de sélection}
\subsubsection{Random}
\begin{figure}[h]
  \begin{center}
    \includegraphics[width=320px]{images/graph-random.png}
  \end{center}
  \caption{random - cross point - bit-flip - replace worst}
\end{figure}

\subsection{Variation de croisement}
\subsubsection{Cross Uniform}
\begin{figure}[h]
  \begin{center}
    \includegraphics[width=320px]{images/graph-cross-uniform.png}
  \end{center}
  \caption{tournament - cross uniform - bit-flip - replace worst}
\end{figure}

\pagebreak
\subsection{Variation de mutation}
\subsubsection{3-Flip}
\begin{figure}[h]
  \begin{center}
    \includegraphics[width=320px]{images/graph-3-flip.png}
  \end{center}
  \caption{tournament - cross point - 3-flip - replace worst}
\end{figure}

\subsubsection{5-Flip}
\begin{figure}[h]
  \begin{center}
    \includegraphics[width=320px]{images/graph-5-flip.png}
  \end{center}
  \caption{tournament - cross point - 5-flip - replace worst}
\end{figure}


\pagebreak
\subsection{Variation d'insertion}
\subsubsection{Age}
\begin{figure}[h]
  \begin{center}
    \includegraphics[width=320px]{images/graph-age.png}
  \end{center}
  \caption{tournament - cross point - bit-flip - age}
\end{figure}



\chapter{Algorithmes autonomes}
\section{Algorithmes considérés}
Sur les 4 algorithmes proposés (roulette adaptative, adaptive pursuit, UCB et roulette uniforme), un seul a été implémenté,
ainsi qu'un autre algorithme issu de mes propres réflexions.

\begin{itemize}
  \item \textbf{Adaptive Pursuit}
  \item \textbf{Algorithme d'automatisation gain / modifications} : Algorithme personalisé, dans lequel la probablité de sélection
	est fonction du gain acquis par rapport au nombre de modifications effectuées (nombre de bits ayant été inversés).
\end{itemize}


\section{Tests et résultats}

Les deux algorithmes testés on un comportement similaire: en début de programme, ils alternent entre les différents types de mutation
avant de fortement favoriser (sans forcement l'utiliser exclusivement) une des deux méthodes pour le reste de l'exécution.


\chapter{Conclusion}

Une variété de comportements, d'approches ont été implémentés pour la première partie du projet. En revanche, par manque de temps
pour l'analyse, la compréhension et l'implémentation d'algorithmes adaptatifs, la deuxième partie a été séverement restreinte. \\ \\

Globalement, la sélection par tournoi, le croisement par point et l'insertion par remplacement des pires instances se sont avérés
être les approches les plus efficaces. Les mutations de type \textit{k-flip} sont plus efficaces pour de petites valeurs de k,
typiquement entre 3 et 5.

\end{document}
